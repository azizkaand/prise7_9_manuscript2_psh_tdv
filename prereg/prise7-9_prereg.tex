% Options for packages loaded elsewhere
\PassOptionsToPackage{unicode}{hyperref}
\PassOptionsToPackage{hyphens}{url}
\PassOptionsToPackage{dvipsnames,svgnames,x11names}{xcolor}
%
\documentclass[
]{article}

\usepackage{amsmath,amssymb}
\usepackage{iftex}
\ifPDFTeX
  \usepackage[T1]{fontenc}
  \usepackage[utf8]{inputenc}
  \usepackage{textcomp} % provide euro and other symbols
\else % if luatex or xetex
  \usepackage{unicode-math}
  \defaultfontfeatures{Scale=MatchLowercase}
  \defaultfontfeatures[\rmfamily]{Ligatures=TeX,Scale=1}
\fi
\usepackage{lmodern}
\ifPDFTeX\else  
    % xetex/luatex font selection
\fi
% Use upquote if available, for straight quotes in verbatim environments
\IfFileExists{upquote.sty}{\usepackage{upquote}}{}
\IfFileExists{microtype.sty}{% use microtype if available
  \usepackage[]{microtype}
  \UseMicrotypeSet[protrusion]{basicmath} % disable protrusion for tt fonts
}{}
\makeatletter
\@ifundefined{KOMAClassName}{% if non-KOMA class
  \IfFileExists{parskip.sty}{%
    \usepackage{parskip}
  }{% else
    \setlength{\parindent}{0pt}
    \setlength{\parskip}{6pt plus 2pt minus 1pt}}
}{% if KOMA class
  \KOMAoptions{parskip=half}}
\makeatother
\usepackage{xcolor}
\setlength{\emergencystretch}{3em} % prevent overfull lines
\setcounter{secnumdepth}{-\maxdimen} % remove section numbering
% Make \paragraph and \subparagraph free-standing
\ifx\paragraph\undefined\else
  \let\oldparagraph\paragraph
  \renewcommand{\paragraph}[1]{\oldparagraph{#1}\mbox{}}
\fi
\ifx\subparagraph\undefined\else
  \let\oldsubparagraph\subparagraph
  \renewcommand{\subparagraph}[1]{\oldsubparagraph{#1}\mbox{}}
\fi


\providecommand{\tightlist}{%
  \setlength{\itemsep}{0pt}\setlength{\parskip}{0pt}}\usepackage{longtable,booktabs,array}
\usepackage{calc} % for calculating minipage widths
% Correct order of tables after \paragraph or \subparagraph
\usepackage{etoolbox}
\makeatletter
\patchcmd\longtable{\par}{\if@noskipsec\mbox{}\fi\par}{}{}
\makeatother
% Allow footnotes in longtable head/foot
\IfFileExists{footnotehyper.sty}{\usepackage{footnotehyper}}{\usepackage{footnote}}
\makesavenoteenv{longtable}
\usepackage{graphicx}
\makeatletter
\def\maxwidth{\ifdim\Gin@nat@width>\linewidth\linewidth\else\Gin@nat@width\fi}
\def\maxheight{\ifdim\Gin@nat@height>\textheight\textheight\else\Gin@nat@height\fi}
\makeatother
% Scale images if necessary, so that they will not overflow the page
% margins by default, and it is still possible to overwrite the defaults
% using explicit options in \includegraphics[width, height, ...]{}
\setkeys{Gin}{width=\maxwidth,height=\maxheight,keepaspectratio}
% Set default figure placement to htbp
\makeatletter
\def\fps@figure{htbp}
\makeatother
% definitions for citeproc citations
\NewDocumentCommand\citeproctext{}{}
\NewDocumentCommand\citeproc{mm}{%
  \begingroup\def\citeproctext{#2}\cite{#1}\endgroup}
\makeatletter
 % allow citations to break across lines
 \let\@cite@ofmt\@firstofone
 % avoid brackets around text for \cite:
 \def\@biblabel#1{}
 \def\@cite#1#2{{#1\if@tempswa , #2\fi}}
\makeatother
\newlength{\cslhangindent}
\setlength{\cslhangindent}{1.5em}
\newlength{\csllabelwidth}
\setlength{\csllabelwidth}{3em}
\newenvironment{CSLReferences}[2] % #1 hanging-indent, #2 entry-spacing
 {\begin{list}{}{%
  \setlength{\itemindent}{0pt}
  \setlength{\leftmargin}{0pt}
  \setlength{\parsep}{0pt}
  % turn on hanging indent if param 1 is 1
  \ifodd #1
   \setlength{\leftmargin}{\cslhangindent}
   \setlength{\itemindent}{-1\cslhangindent}
  \fi
  % set entry spacing
  \setlength{\itemsep}{#2\baselineskip}}}
 {\end{list}}
\usepackage{calc}
\newcommand{\CSLBlock}[1]{\hfill\break\parbox[t]{\linewidth}{\strut\ignorespaces#1\strut}}
\newcommand{\CSLLeftMargin}[1]{\parbox[t]{\csllabelwidth}{\strut#1\strut}}
\newcommand{\CSLRightInline}[1]{\parbox[t]{\linewidth - \csllabelwidth}{\strut#1\strut}}
\newcommand{\CSLIndent}[1]{\hspace{\cslhangindent}#1}

\usepackage[noblocks]{authblk}
\renewcommand*{\Authsep}{, }
\renewcommand*{\Authand}{, }
\renewcommand*{\Authands}{, }
\renewcommand\Affilfont{\small}
\makeatletter
\@ifpackageloaded{caption}{}{\usepackage{caption}}
\AtBeginDocument{%
\ifdefined\contentsname
  \renewcommand*\contentsname{Table of contents}
\else
  \newcommand\contentsname{Table of contents}
\fi
\ifdefined\listfigurename
  \renewcommand*\listfigurename{List of Figures}
\else
  \newcommand\listfigurename{List of Figures}
\fi
\ifdefined\listtablename
  \renewcommand*\listtablename{List of Tables}
\else
  \newcommand\listtablename{List of Tables}
\fi
\ifdefined\figurename
  \renewcommand*\figurename{Figure}
\else
  \newcommand\figurename{Figure}
\fi
\ifdefined\tablename
  \renewcommand*\tablename{Table}
\else
  \newcommand\tablename{Table}
\fi
}
\@ifpackageloaded{float}{}{\usepackage{float}}
\floatstyle{ruled}
\@ifundefined{c@chapter}{\newfloat{codelisting}{h}{lop}}{\newfloat{codelisting}{h}{lop}[chapter]}
\floatname{codelisting}{Listing}
\newcommand*\listoflistings{\listof{codelisting}{List of Listings}}
\makeatother
\makeatletter
\makeatother
\makeatletter
\@ifpackageloaded{caption}{}{\usepackage{caption}}
\@ifpackageloaded{subcaption}{}{\usepackage{subcaption}}
\makeatother
\ifLuaTeX
  \usepackage{selnolig}  % disable illegal ligatures
\fi
\usepackage{bookmark}

\IfFileExists{xurl.sty}{\usepackage{xurl}}{} % add URL line breaks if available
\urlstyle{same} % disable monospaced font for URLs
\hypersetup{
  pdftitle={The Longitudinal Association Between Early Peer Sexual Harassment and Later Dating Violence},
  pdfauthor={Aziz-Kaan Dönmez; Carolina Lunde},
  colorlinks=true,
  linkcolor={blue},
  filecolor={Maroon},
  citecolor={Blue},
  urlcolor={Blue},
  pdfcreator={LaTeX via pandoc}}

\title{The Longitudinal Association Between Early Peer Sexual Harassment
and Later Dating Violence}


\author[1]{Aziz-Kaan Dönmez}
\author[1]{Carolina Lunde}

\affil[1]{University of Gothenburg}


\date{June 19, 2024}
\begin{document}
\maketitle

A preregistration of secondary data analysis on PRISE Data

\subsection{Title}\label{title}

The Longitudinal Association Between Early Peer Sexual Harassment and
Later Dating Violence

\subsection{Description}\label{description}

Sexual harassment (SH) is often conceptualised as unsolicited behaviour
with a sexual nature that can take either verbal or physical forms (Hill
\& Kearl, 2011). It can occur both in-person, for example at school, or
electronically through social media using phones or computers. Studies
on SH among youth have primarily focused on middle-to-high school
students, where roughly 50\% of students in grades 7-12 in American
samples report being victims of SH (Hill \& Kearl, 2011), and similar
frequencies are reported in Swedish samples (Ståhl \& Dennhag, 2021).
Although the prevalence rates of SH are similar for boys and girls, the
types of SH types seem to differ. For example, girls report experiencing
more unwanted comments, touch and gestures whereas boys report
experiencing more homosexual slurs, and visual types (shown or given
unwanted sexual pictures, photos, messages. or notes; Chiodo et al.
(2009).

Revictimization is a term describing the increased risk of being a
victim of one or multiple types of aggression after an initial instance
of one or multiple types of aggressions (Finkelhor et al., 2007). It has
been observed in various situations. Chiodo and colleagues (2009)
observed it among victims of SH, it has also been observed among victims
of child sexual abuse such that individuals are revictimized both in
adolescence (Miron \& Orcutt, 2014) and in adulthood (see C. Classen et
al. (2001)). The rates of revictimisation among victims of child sexual
abuse ranges from 66\% in C. C. Classen et al. (2005) to a mean
prevalence rate of 47.9\% in a recent meta-analysis (Walker et al.,
2019). However, the variability in the point estimate of 47.9\% in
Walker et al. (2019) ranged from 10-90\% and the authors identifies
various reasons for the variability among estimated revictimization
rates across the included studies. Some of the factors are gender, age
and definitions. It is clear that when investigating revictimization one
needs to be cognizant of the contextual factors in which the initial
victimisation occurs.

The aim of this paper is therefore to investigate the longitudinal
association between SH victimisation and subsequent dating violence
victimisation. This is done in the context of the revictimisation theory
which states that individuals who are victims of one or multiple types
of aggressions are more likely to be victims of same or other types of
victimisations in the future. Furthermore, we are interested in whether
there is a dose-dependent relationship between SH victimisation
subsequent dating violence. We would expect a stronger association
between SH victimisation and dating violence victimisation for
individuals who report being sexual harassment at more time points.

\subsection{Hypotheses}\label{hypotheses}

\textbf{H1:} Higher levels of peer-victimization (sexual harassment) at
earlier grades (4, 5, 6, 7 and 7) will predict higher levels of
dating-violence victimization at grade 8.

\textbf{H2:} Previous SH victimisation will be associated with
subsequent SH victimisation.

\textbf{H3:} The association between SH victimisation and TDV
victimisation will be moderated by gender such that the association will
be stronger for girls compared to boys.

\section{Design Plan}\label{design-plan}

\subsection{Study type}\label{study-type}

\textbf{Observational Study}. The study started in 2019 when
participants were in grade 4. It is ongoing and the last wave of data
collection (4th) were conducted in 2023. The next wave (5th) is planned
to be collected in the first half of 2024, at which point the
participants will be in grade 8.

\subsection{Study design}\label{study-design}

The data for this paper will come from the ongoing longitudinal
PRISE-project. In short, the PRISE project recruited participants from
schools around in Västra Götaland, Sweden. The first wave of data
collection was conducted in 2019 and new ones have been collected yearly
since then. The project is, as of right now (2023-11-27), in its 4th
wave with plans to collect the 5th wave in February 2024. It has adopted
a survey methodology where a comprehensive list of variables related to
peer-victimization and its correlates are measured through
questionnaires. The majority of participants fills out the questionnaire
in school with the presence of a research personnel. However, those who
are absentee at the time of data-collection, or those students who are
not in the recruited schools are given the opportunity to fill out the
questionnaire at home.

\section{Sampling Plan}\label{sampling-plan}

\subsection{Existing data}\label{existing-data}

\textbf{Registration following analysis of the data}. As of the date of
submission, some of the data used for this paper has been accessed and
analyzed. However, the analysis conducted has primarily been
psychometric evaluations of the peer sexual harassment scale (PSH-C; one
of the main predictor variables in this paper). This has been done for
T1 (grade 4), T3 (grade 6) and T4 (grade 7). However, the author AKD has
not analysed any associations between the three different time points.
Furthermore, psychometric evaluations have been conducted on the outcome
variable (Dating Violence) for T4, but no associations between the
outcome variable and predictor variables have been conducted prior to
date.

\subsection{Explanation of existing
data}\label{explanation-of-existing-data}

Much of the variables of interest from the existing data has been
psychometrically investigated (code used for these analysis will be made
available in the osf page: https://osf.io/hg4ep/). That is, a
confirmatory factor analysis have been ran on the variables. However, no
cross-sectional or longitudinal associations between the variables have
been investigated. The author conducting the analyses (AKD) have not
been involved with data-collection prior to T4.

\subsection{Data collection
procedures}\label{data-collection-procedures}

Schools around the Västra Götaland region in Sweden were contacted and
asked to participate in the study. The process of selecting schools was
primarily guided by the enrollment patterns of participants in the PRISE
study. In instances where participants transitioned to a different
school for grade 7, schools were chosen based on the destination schools
for the majority of graduating students. This decision was informed by
data obtained from the elementary school management office of the
regional municipality. Essentially, the selection hinged on a
combination of the original schools attended by PRISE participants and,
when applicable, the schools to which most students from the graduating
schools transitioned, as indicated by information from the regional
elementary school management office. A total of 25 schools were
contacted, 11 of which agreed to participate. Guardians of students in
grade 7 were sent information about the study, and consent forms via
post. The schools were then visited, and students whos' guardians had
given consent were given information about the study and then asked to
participate in the study. Students whos' parents had given consent but
were absent from class during the time of data-collection, and students
from the PRISE study that did not attend any of the recruited schools,
were contacted via post and asked to participate in the study online at
home.

\subsection{Sample size and rationale}\label{sample-size-and-rationale}

The original PRISE 4-6 study determined sample size using power
calculations. From Skoog et al. (2019) : ``\emph{The sample size, N=
1000, is based on conventional calculations (Cohen, 1988), aiming for
80\% power, .05 alpha, the ability to detect small effect sizes, and
using more than ten predictors (Miles \& Shevlin, 2001). The size
further accounts for some attrition (10\%) that might occur over the
study period.''}

The sample size of the continuation of the project, that is PRISE 7-9,
is following the same students and has therefore no power calculations
but instead attempted to recruit as many of the participants in PRISE
4-6 as possible.

\section{Variables}\label{variables}

\subsection{Measured variables}\label{measured-variables}

The following variables are measured in the PRISE questionnaire:

Children and Adolescent Social Support Scale (Kerres Malecki \&
Kilpatrick Demary, 2002)

Self-Perceived Popularity (Vanden Abeele et al., 2014)

The Victim Scale (Rigby, 1998)

Children Self-Efficacy Scale (Bandura, 2006)

Conflict in Adolescent Dating Relationships Inventory (Wolfe et al.,
2001)

Measure of Adolescent Relationship Harassment and Abuse (Rothman et al.,
2022)

Revised Conflict Tactics Scale (Straus et al., 1996)

Peer Sexual Harassment Scale - Child (Valik et al., 2023)

Online Sexual Harassment (Gámez-Guadix \& Incera, 2021)

Restricted Freedom of Movement (Calogero et al., 2021)

Pubertal Development Scale (Carskadon \& Acebo, 1993)

The Body Esteem Scale for Adults and Adolescents (Mendelson et al.,
2001)

The Spence Children's Anxiety Scale (Spence, 1998)

Center for Epidemiologic Studies Depression Scale-revised (Haroz et al.,
2014)

Self-Silencing Adolescent Femininity Ideology Scale (Tolman \& Porche,
2000)

Interpersonal Reactivity Index self-report (Davis, 1980)

Dark Triad Dirty Dozen (Jonason \& Webster, 2010)

Problem Behaviour Frequency Scale (Farrell et al., 2016)

Revised Life Orientation Test - Revised (Scheier et al., 1994)

The Emotion Regulation Questionnaire for children \& adolescents
(ERQ-CA) (Gullone \& Taffe, 2012)

The Flourishing Scale (Diener et al., 2009)

The Children's Hope Scale (Snyder et al., 1997)

Objectified Body Consciousness Scale: Youth (Lindberg et al., 2006)

\subsection{Predictor variable}\label{predictor-variable}

\textbf{Peer Sexual Harassment:} Peer SH has been measured (and will be
measured at grade 8) using the Peer Sexual Harassment Child Scale (Valik
et al., 2023). This is a 6 item measure asking participants whether they
have experienced various situations. The items were prefaced with: ``The
following questions are about things that can happen in school against
your will or things that do not feel good. Think about your (current
grade) when rating the items. Remember that we do not disclose your
answers.''. Items were then presented and rated on a scale from: 1=
Never; 2= Once; 3= Few times; 4= Many times

\subsection{Outcome Variables}\label{outcome-variables}

\textbf{Teen Dating Violence:} The Measure of Adolescent Relationship
Harassment and Abuse. (Rothman et al., 2022) was used to measure Teen
Dating Violence. This is a 3 item measure rated as ``YES this has
happened'' or ``NO this has not happened''.

\subsection{Indices}\label{indices}

No indices will be used as we will use Structural Equation Modeling for
our analyses.

\section{Analysis Plan and Statistical
models}\label{analysis-plan-and-statistical-models}

\subsection{Measurement Model
Assessment}\label{measurement-model-assessment}

Analysis 1. Confirmatory factor analyses will be conducted on the listed
measures to confirm its factor structure. Analysis 2. A longitudinal and
gender invariance analysis will be conducted in order to assess whether
the factor structure of the measurements holds over time and gender.

\subsection{H1 and H2}\label{h1-and-h2}

Analysis 3. A Latent Variable Cross-lagged Panel Model will be used to
assess the association between Sexual Harassment in grade 4, 5, 6, 7 and
8, and dating violence in grade 8.

\subsection{H3}\label{h3}

Analysis 4. Gender will be added as a moderator to the same model used
in in Analysis 3 in order to test whether there are significant
differences in strengths of associations between boys and girls

\subsection{Transformations}\label{transformations}

Gender will be dummy-coded such that 1 represents Boys and 2 represents
Girls.

\subsection{Inference criteria}\label{inference-criteria}

Alpha levels will be set to .05, fit indices for the SEM models will be
simulated using our specific factor structure instead of using
conventional cut-off values.

\subsection{Data exclusion}\label{data-exclusion}

Since our outcome variable is Teen Dating Violence, only participants
who report having been in romantic relationships (defined as
relationships or dating) will be included.

\subsection{Missing data}\label{missing-data}

Multiple imputation will be performed using the Multivariate Imputation
by Chained Equations (mice) package in R (Van Buuren \&
Groothuis-Oudshoorn, 2011) . General victimisation (measured by The
Victim Scale; Rigby (1998) and online sexual harassment victimisation
(Gámez-Guadix \& Incera, 2021) will be used as proxy-variables during
the imputation.

\section{Other}\label{other}

Link to osf page: https://osf.io/hg4ep/

\section*{References}\label{references}
\addcontentsline{toc}{section}{References}

\phantomsection\label{refs}
\begin{CSLReferences}{1}{0}
\bibitem[\citeproctext]{ref-bandura2006}
Bandura, A. (2006). Guide for constructing self-efficacy scales.
\emph{Self-Efficacy Beliefs of Adolescents}, \emph{5}(1), 307--337.

\bibitem[\citeproctext]{ref-calogero2021}
Calogero, R. M., Tylka, T. L., Siegel, J. A., Pina, A., \& Roberts,
T.-A. (2021). Smile pretty and watch your back: Personal safety anxiety
and vigilance in objectification theory. \emph{Journal of Personality
and Social Psychology}, \emph{121}(6), 1195--1222.
\url{https://doi.org/10.1037/pspi0000344}

\bibitem[\citeproctext]{ref-carskadon1993}
Carskadon, M. A., \& Acebo, C. (1993). A self-administered rating scale
for pubertal development. \emph{Journal of Adolescent Health},
\emph{14}(3), 190--195.

\bibitem[\citeproctext]{ref-chiodo2009a}
Chiodo, D., Wolfe, D. A., Crooks, C., Hughes, R., \& Jaffe, P. (2009).
Impact of Sexual Harassment Victimization by Peers on Subsequent
Adolescent Victimization and Adjustment: A Longitudinal Study.
\emph{Journal of Adolescent Health}, \emph{45}(3), 246--252.
\url{https://doi.org/10.1016/j.jadohealth.2009.01.006}

\bibitem[\citeproctext]{ref-classen2005}
Classen, C. C., Palesh, O. G., \& Aggarwal, R. (2005). Sexual
revictimization: A review of the empirical literature. \emph{Trauma,
Violence, \& Abuse}, \emph{6}(2), 103--129.

\bibitem[\citeproctext]{ref-classen2001}
Classen, C., Field, N. P., Koopman, C., Nevill-Manning, K., \& Spiegel,
D. (2001). Interpersonal problems and their relationship to sexual
revictimization among women sexually abused in childhood. \emph{Journal
of Interpersonal Violence}, \emph{16}(6), 495--509.

\bibitem[\citeproctext]{ref-cohen1988}
Cohen, S. (1988). \emph{Perceived stress in a probability sample of the
united states.}

\bibitem[\citeproctext]{ref-davis1980}
Davis, M. H. (1980). \emph{Interpersonal reactivity index}.

\bibitem[\citeproctext]{ref-diener2009}
Diener, E., Wirtz, D., Biswas-Diener, R., Tov, W., Kim-Prieto, C., Choi,
D., \& Oishi, S. (2009). New measures of well-being. \emph{Assessing
Well-Being: The Collected Works of Ed Diener}, 247--266.

\bibitem[\citeproctext]{ref-farrell2016}
Farrell, A. D., Sullivan, T. N., Goncy, E. A., \& Le, A.-T. H. (2016).
Assessment of adolescents{'} victimization, aggression, and problem
behaviors: Evaluation of the problem behavior frequency scale.
\emph{Psychological Assessment}, \emph{28}(6), 702.

\bibitem[\citeproctext]{ref-finkelhor2007}
Finkelhor, D., Ormrod, R. K., \& Turner, H. A. (2007).
Poly-victimization: A neglected component in child victimization.
\emph{Child Abuse \& Neglect}, \emph{31}(1), 7--26.

\bibitem[\citeproctext]{ref-guxe1mez-guadix2021}
Gámez-Guadix, M., \& Incera, D. (2021). Homophobia is online: Sexual
victimization and risks on the internet and mental health among
bisexual, homosexual, pansexual, asexual, and queer adolescents.
\emph{Computers in Human Behavior}, \emph{119}, 106728.
\url{https://doi.org/10.1016/j.chb.2021.106728}

\bibitem[\citeproctext]{ref-gullone2012}
Gullone, E., \& Taffe, J. (2012). The emotion regulation questionnaire
for children and adolescents (ERQ{\textendash}CA): A psychometric
evaluation. \emph{Psychological Assessment}, \emph{24}(2), 409.

\bibitem[\citeproctext]{ref-haroz2014}
Haroz, E. E., Ybarra, M. L., \& Eaton, W. W. (2014). Psychometric
evaluation of a self-report scale to measure adolescent depression: The
CESDR-10 in two national adolescent samples in the united states.
\emph{Journal of Affective Disorders}, \emph{158}, 154--160.

\bibitem[\citeproctext]{ref-hill2011}
Hill, C., \& Kearl, H. (2011). \emph{Crossing the line: Sexual
harassment at school.} ERIC.

\bibitem[\citeproctext]{ref-jonason2010}
Jonason, P. K., \& Webster, G. D. (2010). The dirty dozen: A concise
measure of the dark triad. \emph{Psychological Assessment},
\emph{22}(2), 420.

\bibitem[\citeproctext]{ref-kerresmalecki2002}
Kerres Malecki, C., \& Kilpatrick Demary, M. (2002). Measuring perceived
social support: Development of the child and adolescent social support
scale (CASSS). \emph{Psychology in the Schools}, \emph{39}(1), 1--18.

\bibitem[\citeproctext]{ref-lindberg2006}
Lindberg, S. M., Hyde, J. S., \& McKinley, N. M. (2006). A measure of
objectified body consciousness for preadolescent and adolescent youth.
\emph{Psychology of Women Quarterly}, \emph{30}(1), 65--76.

\bibitem[\citeproctext]{ref-mendelson2001}
Mendelson, B. K., Mendelson, M. J., \& White, D. R. (2001). Body-esteem
scale for adolescents and adults. \emph{Journal of Personality
Assessment}, \emph{76}(1), 90--106.

\bibitem[\citeproctext]{ref-miles2001}
Miles, J., \& Shevlin, M. (2001). \emph{Applying regression and
correlation: A guide for students and researchers}. Sage.

\bibitem[\citeproctext]{ref-miron2014}
Miron, L. R., \& Orcutt, H. K. (2014). Pathways from childhood abuse to
prospective revictimization: Depression, sex to reduce negative affect,
and forecasted sexual behavior. \emph{Child Abuse \& Neglect},
\emph{38}(11), 1848--1859.

\bibitem[\citeproctext]{ref-rigby1998}
Rigby, K. (1998). The relationship between reported health and
involvement in bully/victim problems among male and female secondary
schoolchildren. \emph{Journal of Health Psychology}, \emph{3}(4),
465--476.

\bibitem[\citeproctext]{ref-rothman2022}
Rothman, E. F., Cuevas, C. A., Mumford, E. A., Bahrami, E., \& Taylor,
B. G. (2022). The psychometric properties of the measure of adolescent
relationship harassment and abuse (MARSHA) with a nationally
representative sample of US youth. \emph{Journal of Interpersonal
Violence}, \emph{37}(11-12), NP9712--NP9737.

\bibitem[\citeproctext]{ref-scheier1994}
Scheier, M. F., Carver, C. S., \& Bridges, M. W. (1994). Distinguishing
optimism from neuroticism (and trait anxiety, self-mastery, and
self-esteem): A reevaluation of the life orientation test. \emph{Journal
of Personality and Social Psychology}, \emph{67}(6), 1063.

\bibitem[\citeproctext]{ref-skoog2019}
Skoog, T., Holmqvist Gattario, K., \& Lunde, C. (2019). Study protocol
for PRISE: A longitudinal study of sexual harassment during the
transition from childhood to adolescence. \emph{BMC Psychology},
\emph{7}, 1--10.

\bibitem[\citeproctext]{ref-snyder1997}
Snyder, C. R., Hoza, B., Pelham, W. E., Rapoff, M., Ware, L., Danovsky,
M., Highberger, L., Ribinstein, H., \& Stahl, K. J. (1997). The
development and validation of the children{'}s hope scale. \emph{Journal
of Pediatric Psychology}, \emph{22}(3), 399--421.

\bibitem[\citeproctext]{ref-spence1998}
Spence, S. H. (1998). A measure of anxiety symptoms among children.
\emph{Behaviour Research and Therapy}, \emph{36}(5), 545--566.
\url{https://doi.org/10.1016/S0005-7967(98)00034-5}

\bibitem[\citeproctext]{ref-stuxe5hl2021}
Ståhl, S., \& Dennhag, I. (2021). Online and offline sexual harassment
associations of anxiety and depression in an adolescent sample.
\emph{Nordic Journal of Psychiatry}, \emph{75}(5), 330--335.
\url{https://doi.org/10.1080/08039488.2020.1856924}

\bibitem[\citeproctext]{ref-straus1996}
Straus, M. A., Hamby, S. L., Boney-McCoy, S. U. E., \& Sugarman, D. B.
(1996). The revised conflict tactics scales (CTS2) development and
preliminary psychometric data. \emph{Journal of Family Issues},
\emph{17}(3), 283--316.

\bibitem[\citeproctext]{ref-tolman2000}
Tolman, D. L., \& Porche, M. V. (2000). The adolescent femininity
ideology scale: Development and validation of a new measure for girls.
\emph{Psychology of Women Quarterly}, \emph{24}(4), 365--376.

\bibitem[\citeproctext]{ref-valik2023}
Valik, A., Holmqvist Gattario, K., Lunde, C., \& Skoog, T. (2023).
PSH-C: A measure of peer sexual harassment among children. \emph{Journal
of Social Issues}, \emph{79}(4), 1123--1146.
\url{https://doi.org/10.1111/josi.12517}

\bibitem[\citeproctext]{ref-vanbuuren2011}
Van Buuren, S., \& Groothuis-Oudshoorn, K. (2011). Mice: Multivariate
imputation by chained equations in r. \emph{Journal of Statistical
Software}, \emph{45}, 1--67.

\bibitem[\citeproctext]{ref-vandenabeele2014}
Vanden Abeele, M., Campbell, S. W., Eggermont, S., \& Roe, K. (2014).
Sexting, mobile porn use, and peer group dynamics: Boys' and girls'
self-perceived popularity, need for popularity, and perceived peer
pressure. \emph{Media Psychology}, \emph{17}(1), 6--33.

\bibitem[\citeproctext]{ref-walker2019}
Walker, H. E., Freud, J. S., Ellis, R. A., Fraine, S. M., \& Wilson, L.
C. (2019). The prevalence of sexual revictimization: A meta-analytic
review. \emph{Trauma, Violence, \& Abuse}, \emph{20}(1), 67--80.

\bibitem[\citeproctext]{ref-wolfe2001}
Wolfe, D. A., Scott, K., Reitzel-Jaffe, D., Wekerle, C., Grasley, C., \&
Straatman, A.-L. (2001). Development and validation of the conflict in
adolescent dating relationships inventory. \emph{Psychological
Assessment}, \emph{13}(2), 277.

\end{CSLReferences}



\end{document}
